\documentclass[12pt]{article}

\usepackage{fullpage}
\usepackage{multicol,multirow}
\usepackage{tabularx}
\usepackage{listings}
\usepackage{pgfplots}
\usepackage[utf8]{inputenc}
\usepackage[russian]{babel}
\usepackage{pgfplots}
\usepackage{tikz}

% Оригиналный шаблон: http://k806.ru/dalabs/da-report-template-2012.tex

\begin{document}

\section*{Лабораторная работа №9\, по курсу дискрeтного анализа: Графы}

Выполнил студент группы М8О-312Б-22 МАИ \textit{Корнев Максим}.

\subsection*{Условие}
\textbf{Вариант:} 6. Поиск кратчайших путей между всеми парами вершин алгоритмом Джонсона

Задан взвешенный ориентированный граф, состоящий из n вершин и m ребер. Вершины пронумерованы целыми числами от 1 до n. Необходимо найти длины кратчайших путей между всеми парами вершин при помощи алгоритма Джонсона. Длина пути равна сумме весов ребер на этом пути. Обратите внимание, что в данном варианте веса ребер могут быть отрицательными, поскольку алгоритм умеет с ними работать. Граф не содержит петель и кратных ребер.


\newpage
\section*{Метод решения}

\begin{enumerate}
    \item Так как алгоритм Дейкстры не умеет работать с отрицательными рёбрами, необходимо на время избавиться от них в нашем графе. Для этого мы добавляем в граф фиктивную вершину \( S \) и строим из неё рёбра с весом \( 0 \) в каждую вершину исходного графа.
    
    \item Для нового графа запускаем алгоритм Беллмана -- Форда, который либо обнаруживает наличие отрицательного цикла в графе и завершает алгоритм, либо возвращает кратчайшие расстояния от фиктивной вершины \( S \) до каждой вершины исходного графа. Суть алгоритма заключается в том, что мы \( V - 1 \) раз проходим по всем рёбрам и релаксируем их, если 
    \[
    d[v] > d[u] + w(u, w).
    \]
    Если на \( V \)-ой итерации происходит ещё одна релаксация, то в графе имеется отрицательный цикл. С помощью этих кратчайших расстояний мы перевзвешиваем рёбра по следующей формуле:
    \[
    \omega\varphi(u, v) = \omega(u, v) + \varphi(u) - \varphi(v).
    \]
    Удаляем фиктивную вершину и запускаем алгоритм Дейкстры для каждой вершины графа, который возвращает кратчайшие расстояния до каждой другой вершины графа. Для преобразования этих расстояний к изначальному графу необходимо применить обратную формулу перевзвешивания:
    \[
    \omega\varphi(u, v) = \omega(u, v) - \varphi(u) + \varphi(v).
    \]
    
    \item Суть алгоритма Дейкстры заключается в том, что в алгоритме поддерживается множество вершин, для которых уже вычислены длины кратчайших путей до них из \( s \). На каждой итерации основного цикла выбирается вершина, не помеченная посещённой, которой на текущий момент соответствует минимальная оценка кратчайшего пути. Вершина добавляется в множество посещённых и производится релаксация всех исходящих из неё рёбер.
\end{enumerate}


\newpage
\subsection*{Описание программы}

Для реализации алгоритма были реализованы следующие функции:

\begin{enumerate}
    \item \textbf{BellmanFord(Graph* graph)}:
    \begin{itemize}
        \item Выполняет алгоритм Беллмана-Форда для поиска кратчайших расстояний до всех вершин графа.
        \item Проверяет наличие отрицательных циклов в графе:
        \[
        \text{Если } d[\text{from}] + \text{cost} < d[\text{to}], \text{ то цикл отрицательный.}
        \]
        \item Перевзвешивает рёбра, корректируя их стоимости с учётом расстояний.

    \end{itemize}
    
    \item \textbf{Dijkstra(const Graph* graph, const long long* baseDistances)}:
    \begin{itemize}
        \item Для каждой вершины выполняет алгоритм Дейкстры, начиная с неё как стартовой.
        \item Использует массив расстояний и массив посещённых вершин для поиска кратчайших путей.
        \item В каждой итерации выбирает вершину с минимальным текущим расстоянием.
        \item Обновляет расстояния для соседних вершин через рёбра графа.
        \item Корректирует финальные расстояния с учётом базовых расстояний:
        \[
        d[i] = \text{distances}[i] - \text{baseDistances}[\text{startVertex}] + \text{baseDistances}[i].
        \]
    \end{itemize}
\end{enumerate}


\newpage
\subsection*{Дневник отладки}

\begin{enumerate}
    \item Получил WA на 1 тесте, так как перепутал знаки в формуле Беллмана-Форда.
\end{enumerate}

\newpage
\subsection*{Тест производительности}

Алгоритм работает за время $O(V^3 + VE)$, где V - количество вершин E - количество ребер

\begin{tikzpicture}

\begin{axis}[xlabel={количество вершин}, ylabel={Время, с}]
\addplot coordinates {
    (100, 0.005)
    (1000, 2.268)
    (2000, 3.699)
    (5000, 6.123)
};
\end{axis}
\end{tikzpicture}


\newpage
\section*{Выводы}
В ходе выполнения лабораторной работы я изучил алгоритм Джонсона и применил его для решения задачи нахождения кратчайших путей между всеми парами вершин в графе. Полученные знания позволили успешно реализовать и проверить работу алгоритма на практике.


\end{document}


