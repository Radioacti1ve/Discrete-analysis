\documentclass[12pt]{article}

\usepackage{fullpage}
\usepackage{multicol,multirow}
\usepackage{tabularx}
\usepackage{listings}
\usepackage{pgfplots}
\usepackage[utf8]{inputenc}
\usepackage[russian]{babel}
\usepackage{pgfplots}
\usepackage{tikz}

% Оригиналный шаблон: http://k806.ru/dalabs/da-report-template-2012.tex

\begin{document}

\section*{Лабораторная работа №\,8 по курсу дискрeтного анализа: 
Динамическое программирование}

Выполнил студент группы М8О-312Б-22 МАИ \textit{Корнев Максим}.

\subsection*{Условие}
При помощи метода динамического программирования разработать алгоритм 
решения задачи, определяемой своим вариантом; оценить время выполнения 
алгоритма и объем затрачиваемой оперативной памяти. Перед выполнением 
задания необходимо обосновать применимость метода динамического 
программирования.
\textbf{Вариант 1: Хитрый рюкзак.} У вас есть рюкзак, вместимостью 
$m$, а также $n$ предметов, у каждого из которых есть вес $w_i$ и 
стоимость $c_i$. Необходимо выбрать такое подмножество $I$ из них, чтобы
\begin{itemize}
\item $\sum\limits_{i \in I} w_i \leq m$
\item $(\sum\limits_{i \in I} c_i) \ast |I|$ является максимальной из всех 
возможных
\end{itemize}
$|I|$ — мощность множества $I$.


\newpage
\subsection*{Метод решения}

 Для решения воспользуемся методом динамического программирования, в котором 
большая задача разбивается на множество маленьких подзадач, решения каждой 
из которых может быть использовано несколько раз, что сильно увеличивает 
производительность алгоритма. Вспомним решение изветсной задачи о рюкзаке: 
составим таблицу из $n$ строк и $m$ столбцов и заполним каждую ячейку по 
определённому правилу, причём в заполнении каждой ячейки будут участвовать 
уже заполненные ячейки, в итоге в правой нижней ячейке нашей таблицы будет 
лежать решение задачи, то есть максимальная стоимость, которую можно унести 
в рюкзаке. Задача из моего варианта отличается от оригинальной условием 
заполнения таблицы, а также тем, что нужно хранить вспомогательную таблицу 
для восстановления последовательности предметов.

% \newpage
\subsection*{Описание программы}

Программа состоит из одного файла.


\newpage
\subsection*{Дневник отладки}

\begin{enumerate}
    \item Программа выдала {OK} с 1 попытки.
\end{enumerate}

\newpage
\subsection*{Тест производительности}

    Ниже приведен тест времени работы алгоритма. По оси $X$ — количество 
    предметов, по оси $Y$ — время выполнения алгоритма в мс (меньше — лучше).
    
    \begin{tikzpicture}
        \begin{axis} [
            ymin = 0
        ]
        \addplot coordinates {
            (50, 33) 
            (100,51) 
            (250,749) 
            (500,8391)
        };
        \end{axis}
    \end{tikzpicture}

    Тесты подтвердили временную сложность алгоритма — $O(n^2m)$



\newpage
\subsection*{Выводы}

Проделав лабораторную работу, я ознакомился с новым подходом решения 
алгоритмических задач — динамическим программированием - и смог сделать довольно популярную задачу о рюкзаке
\end{document}
