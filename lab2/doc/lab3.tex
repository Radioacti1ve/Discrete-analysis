\documentclass[12pt]{article}

\usepackage{fullpage}
\usepackage{multicol,multirow}
\usepackage{tabularx}
\usepackage{listings}
\usepackage{pgfplots}
\usepackage[utf8]{inputenc}
\usepackage[russian]{babel}

% Оригиналный шаблон: http://k806.ru/dalabs/da-report-template-2012.tex

\begin{document}

\section*{Лабораторная работа №3\, по курсу дискрeтного анализа: Исследование качества программ}

Выполнил студент группы М8О-212Б-22 МАИ \textit{Корнев Максим}.

\subsection*{Условие}


Для реализации словаря из предыдущей лабораторной работы, необходимо провести исследование скорости выполнения и потребления оперативной памяти.
В случае выявления ошибок или явных недочётов, требуется их исправить.

Результатом лабораторной работы является отчёт, состоящий из:

\begin{itemize}
    \item Дневника выполнения работы, в котором отражено что и когда делалось, какие средства использовались и какие результаты были достигнуты на каждом шаге выполнения
    лабораторной работы.
    \item Выводов о найденных недочётах.
    \item Сравнение работы исправленной программы с предыдущей версией.
    \item Общих выводов о выполнении лабораторной работы, полученном опыте.
\end{itemize}

Минимальный набор используемых средств должен содержать утилиту gprof и библиотеку dmalloc,
однако их можно заменять на любые другие аналогичные или более развитые утилиты (например, Valgrind или Shark)
или добавлять к ним новые (например, gcov).

\newpage
\subsection*{Метод решения}

В рамках выполнения лабораторной работы я буду использовать следующие утилиты:
\begin{itemize}
    \item Анализ времени работы: gprof
    \item Анализ потребления памяти: valgrind
\end{itemize}

% \newpage
\subsection*{gprof}

Утилита gprof позволяет измерить время работы всех функций в программе, 
а также количество их вызовов и долю от общего времени работы программы в процентах.

Для работы с утилитой gprof было необходимо скомпилировать программу с ключом \texttt{-pg}
После запуска полученного исполняемого файла появился файл gmon.out, в котом содержалась информация
предоставленная для анализа программы. Далее этот файл был обработан gprof для получения текстового
файла с подробной информацией о времени работы и вызовах всех функций и операторов,
которые использовались в программе.
Вывод программы:
\begin{lstlisting}

Flat profile:

Each sample counts as 0.01 seconds.
  %   cumulative   self              self     total           
 time   seconds   seconds    calls  ms/call  ms/call  name    
 28.57      0.02     0.02   526864     0.00     0.00  AVLTree::Search(std::__cxx11::basic_string<char, std::char_traits<char>, std::allocator<char> > const&)
 14.29      0.03     0.01        1    10.00    10.00  Node::InsertNode(AVLTree*, std::pair<std::__cxx11::basic_string<char, std::char_traits<char>, std::allocator<char> >, unsigned long long> const&, bool&)
 14.29      0.04     0.01                             Node::Print(int)
  0.00      0.04     0.00      216     0.00     0.00  _init
  0.00      0.04     0.00       44     0.00     0.00  AVLTree::InsertNode(std::pair<std::__cxx11::basic_string<char, std::char_traits<char>, std::allocator<char> >, unsigned long long>&)

 %         the percentage of the total running time of the
time       program used by this function.

cumulative a running sum of the number of seconds accounted
 seconds   for by this function and those listed above it.

 self      the number of seconds accounted for by this
seconds    function alone.  This is the major sort for this
           listing.

calls      the number of times this function was invoked, if
           this function is profiled, else blank.

 self      the average number of milliseconds spent in this
ms/call    function per call, if this function is profiled,
     else blank.

 total     the average number of milliseconds spent in this
ms/call    function and its descendents per call, if this
     function is profiled, else blank.

name       the name of the function.  This is the minor sort
           for this listing. The index shows the location of
     the function in the gprof listing. If the index is
     in parenthesis it shows where it would appear in
     the gprof listing if it were to be printed.


Copyright (C) 2012-2022 Free Software Foundation, Inc.

Copying and distribution of this file, with or without modification,
are permitted in any medium without royalty provided the copyright
notice and this notice are preserved.


         Call graph (explanation follows)


granularity: each sample hit covers 4 byte(s) for 25.00% of 0.04 seconds


index % time    self  children    called     name
                0.00    0.00   33458/526864      Node::Print(int) [4]
                0.00    0.00   66542/526864      AVLTree::Load(std::basic_ifstream<char, std::char_traits<char> >&) [3]
                0.02    0.00  426864/526864      AVLTree::Clear() [2]
[1]     50.0    0.02    0.00  526864         AVLTree::Search(std::__cxx11::basic_string<char, std::char_traits<char>, std::allocator<char> > const&) [1]
-----------------------------------------------
                                                 <spontaneous>
[2]     40.5    0.00    0.02                 AVLTree::Clear() [2]
                0.02    0.00  426864/526864      AVLTree::Search(std::__cxx11::basic_string<char, std::char_traits<char>, std::allocator<char> > const&) [1]
-----------------------------------------------
                                                 <spontaneous>
[3]     31.3    0.00    0.01                 AVLTree::Load(std::basic_ifstream<char, std::char_traits<char> >&) [3]
                0.01    0.00       1/1           Node::InsertNode(AVLTree*, std::pair<std::__cxx11::basic_string<char, std::char_traits<char>, std::allocator<char> >, unsigned long long> const&, bool&) [5]
                0.00    0.00   66542/526864      AVLTree::Search(std::__cxx11::basic_string<char, std::char_traits<char>, std::allocator<char> > const&) [1]
-----------------------------------------------
                                                 <spontaneous>
[4]     28.2    0.01    0.00                 Node::Print(int) [4]
                0.00    0.00   33458/526864      AVLTree::Search(std::__cxx11::basic_string<char, std::char_traits<char>, std::allocator<char> > const&) [1]
-----------------------------------------------
                                5434             Node::InsertNode(AVLTree*, std::pair<std::__cxx11::basic_string<char, std::char_traits<char>, std::allocator<char> >, unsigned long long> const&, bool&) [5]
                0.01    0.00       1/1           AVLTree::Load(std::basic_ifstream<char, std::char_traits<char> >&) [3]
[5]     25.0    0.01    0.00       1+5434    Node::InsertNode(AVLTree*, std::pair<std::__cxx11::basic_string<char, std::char_traits<char>, std::allocator<char> >, unsigned long long> const&, bool&) [5]
                                5434             Node::InsertNode(AVLTree*, std::pair<std::__cxx11::basic_string<char, std::char_traits<char>, std::allocator<char> >, unsigned long long> const&, bool&) [5]
-----------------------------------------------
                                1270             _init [11]
                0.00    0.00     216/216         Node::Node(std::pair<std::__cxx11::basic_string<char, std::char_traits<char>, std::allocator<char> >, unsigned long long>, Node*) [24]
[11]     0.0    0.00    0.00     216+1270    _init [11]
                0.00    0.00      44/44          AVLTree::InsertNode(std::pair<std::__cxx11::basic_string<char, std::char_traits<char>, std::allocator<char> >, unsigned long long>&) [12]
                                1270             _init [11]
-----------------------------------------------
                0.00    0.00      44/44          _init [11]
[12]     0.0    0.00    0.00      44         AVLTree::InsertNode(std::pair<std::__cxx11::basic_string<char, std::char_traits<char>, std::allocator<char> >, unsigned long long>&) [12]
-----------------------------------------------

 This table describes the call tree of the program, and was sorted by
 the total amount of time spent in each function and its children.

 Each entry in this table consists of several lines.  The line with the
 index number at the left hand margin lists the current function.
 The lines above it list the functions that called this function,
 and the lines below it list the functions this one called.
 This line lists:
     index  A unique number given to each element of the table.
    Index numbers are sorted numerically.
    The index number is printed next to every function name so
    it is easier to look up where the function is in the table.


     % time  This is the percentage of the `total' time that was spent
    in this function and its children.  Note that due to
    different viewpoints, functions excluded by options, etc,
    these numbers will NOT add up to 100%.

     self  This is the total amount of time spent in this function.

     children  This is the total amount of time propagated into this
    function by its children.

     called  This is the number of times the function was called.
    If the function called itself recursively, the number
    only includes non-recursive calls, and is followed by
    a `+' and the number of recursive calls.

     name  The name of the current function.  The index number is
    printed after it.  If the function is a member of a
    cycle, the cycle number is printed between the
    function's name and the index number.


 For the function's parents, the fields have the following meanings:

     self  This is the amount of time that was propagated directly
    from the function into this parent.

     children  This is the amount of time that was propagated from
    the function's children into this parent.

     called  This is the number of times this parent called the
    function `/' the total number of times the function
    was called.  Recursive calls to the function are not
    included in the number after the `/'.

     name  This is the name of the parent.  The parent's index
    number is printed after it.  If the parent is a
    member of a cycle, the cycle number is printed between
    the name and the index number.

 If the parents of the function cannot be determined, the word
 `<spontaneous>' is printed in the `name' field, and all the other
 fields are blank.

 For the function's children, the fields have the following meanings:

     self  This is the amount of time that was propagated directly
    from the child into the function.

     children  This is the amount of time that was propagated from the
    child's children to the function.

     called  This is the number of times the function called
    this child `/' the total number of times the child
    was called.  Recursive calls by the child are not
    listed in the number after the `/'.

     name  This is the name of the child.  The child's index
    number is printed after it.  If the child is a
    member of a cycle, the cycle number is printed
    between the name and the index number.

 If there are any cycles (circles) in the call graph, there is an
 entry for the cycle-as-a-whole.  This entry shows who called the
 cycle (as parents) and the members of the cycle (as children.)
 The `+' recursive calls entry shows the number of function calls that
 were internal to the cycle, and the calls entry for each member shows,
 for that member, how many times it was called from other members of
 the cycle.


Copyright (C) 2012-2022 Free Software Foundation, Inc.

Copying and distribution of this file, with or without modification,
are permitted in any medium without royalty provided the copyright
notice and this notice are preserved.


Index by function name

   [5] Node::InsertNode(AVLTree*, std::pair<std::__cxx11::basic_string<char, std::char_traits<char>, std::allocator<char> >, unsigned long long> const&, bool&) [12] AVLTree::InsertNode(std::pair<std::__cxx11::basic_string<char, std::char_traits<char>, std::allocator<char> >, unsigned long long>&) [11] _init
   [4] Node::Print(int)        [1] AVLTree::Search(std::__cxx11::basic_string<char, std::char_traits<char>, std::allocator<char> > const&)

\end{lstlisting}
\texttt{
Операции с AVL-деревьями в графе вызовов указывают на конкретные методы, используемые для манипуляции AVL-деревьями, такие как InsertNode, RemoveNode, LeftRotate и RightRotate. Это дает лучшее понимание того, как реализована структура AVL-дерева в коде. Рекурсивные вызовы также указаны в графе вызовов, хотя их влияние на производительность не всегда очевидно из-за отсутствия информации о времени. Из этих данных можно сделать вывод, что программа включает существенные манипуляции со строками и преобразования данных. Лямбда-функции и рекурсивные элементы указывают на определенный уровень сложности в структуре программы, с возможным использованием функциональных стилей программирования.
Несмотря на подробные профилировочные данные, явные проблемы с производительностью не обнаружены. Это может означать, что программа либо эффективна, либо профилировка не смогла выявить наиболее трудоемкие операции.
}
\subsection*{Valgrind}

Valgrind является утилитой для поиска ошибок в работе с памятью в программе, таких
как утечки памяти и выход за границу массива.
Результат работы \texttt{Valgrind}: \\
\texttt{
    ==47438== Memcheck, a memory error detector\\
==47438== Copyright (C) 2002-2017, and GNU GPL'd, by Julian Seward et al.\\   
==47438== Using Valgrind-3.18.1 and LibVEX; rerun with -h for copyright info\\
==47438== Command: ./main\\
==47438== HEAP SUMMARY:\\
==47438==     in use at exit: 122,880 bytes in 6 blocks\\
==47438==   total heap usage: 15 allocs, 9 frees, 196,818 bytes allocated\\
==47438==\\
==47438== LEAK SUMMARY:\\
==47438==    definitely lost: 0 bytes in 0 blocks\\
==47438==    indirectly lost: 0 bytes in 0 blocks\\
==47438==      possibly lost: 0 bytes in 0 blocks\\
==47438==    still reachable: 122,880 bytes in 6 blocks\\
==47438==         suppressed: 0 bytes in 0 blocks\\
==47438== Rerun with --leak-check=full to see details of leaked memory\\
==47438==\\
==47438== For lists of detected and suppressed errors, rerun with: -s\\
==47438== ERROR SUMMARY: 0 errors from 0 contexts (suppressed: 0 from 0)\\
}

Сообщение \texttt{still reachable: 122,880 bytes in 6 blocks} происходит из-за использования \texttt{ios::sync\_with\_stdio()}.
Я его использовал для ускорения работы программы и это не является утечкой памяти. В результате использования утилиты Valgrind утечки памяти не выявлены — программа работает с памятью корректно.

\newpage
\subsection*{Выводы}
В ходе выполнения практической работы я овладел методами выявления узких мест и проблем в управлении памятью. 
Использование утилиты gprof для анализа времени выполнения программы помогает выявить функции, затрачивающие больше всего времени, что, в свою очередь, способствует оптимизации кода с целью улучшения производительности. 
Применение valgrind для анализа работы с памятью позволяет выявить утечки памяти, неправильное использование указателей и другие проблемы, которые могут привести к непредсказуемому поведению программы.
\end{document}