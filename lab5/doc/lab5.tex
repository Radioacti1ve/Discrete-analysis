\documentclass[12pt]{article}

\usepackage{fullpage}
\usepackage{multicol,multirow}
\usepackage{tabularx}
\usepackage{listings}
\usepackage{pgfplots}
\usepackage[utf8]{inputenc}
\usepackage[russian]{babel}
\usepackage{pgfplots}
\usepackage{tikz}

% Оригиналный шаблон: http://k806.ru/dalabs/da-report-template-2012.tex

\begin{document}

\section*{Лабораторная работа №5\, по курсу дискрeтного анализа: Суффиксные деревья}

Выполнил студент группы М8О-312Б-22 МАИ \textit{Корнев Максим}.

\subsection*{Условие}

\textbf{Вариант:} 2

Найти в заранее известном тексте поступающие на вход образцы с использованием суффиксного массива.

\newpage
\subsection*{Метод решения}

Я решил задачу построения суффиксного дерева с помощью алгоритма Укконена, чтобы эффективно искать подстроки в строке. Сначала реализовал класс для узлов дерева, потом суффиксное дерево, где шаг за шагом добавлял символы строки. Далее на основе дерева построил суффиксный массив и применил бинарный поиск для нахождения всех вхождений подстроки.

% \newpage
\subsection*{Описание программы}

Для реализации алгоритма были реализованы следующие функции и структуры:
\begin{itemize}
    \item \texttt{class Node}: представляет узел суффиксного дерева. Хранит ссылки на дочерние узлы, указатели на начало и конец строки, а также ссылку на суффиксную свя
    \item \texttt{class SufTree}: Реализует суффиксное дерево с помощью алгоритма Укконена. Постепенно расширяет дерево, добавляя символы строки, и поддерживает активную точку (узел, длину и активное ребро). Содержит методы для построения дерева, управления суффиксными ссылками и поиска.
    \item \texttt{class SufArr}: построен на основе суффиксного дерева. Хранит суффиксный массив и предоставляет метод Search для поиска всех вхождений подстроки с использованием бинарного поиска {equal\_range}.
\end{itemize}

\newpage
\subsection*{Дневник отладки}

\begin{enumerate}
    \item Был получен TL на тесте №1. Немного напутал в логике функции суффиксного дерева. 
    \item Был получен WA на тесте №1. При поиске в суффиксном массиве когда добавлял левую границу случайно добавлял 1, а не l.
\end{enumerate}

\newpage
\subsection*{Тест производительности}

Алгоритм работает за время $O(n + m logn)$, где n - длина первой строки, m - длина паттерна.

\begin{tikzpicture}
\begin{axis}[xlabel={Время, s}, ylabel={Длина строки, символы}]
\addplot coordinates {
    (0.0039, 1000)
    (0.0082, 5000)
    (0.0184, 10000)
    (0.0423, 20000)
    (0.3517, 50000)
    (0.72629, 100000)
};
\end{axis}
\end{tikzpicture}


\newpage
\subsection*{Выводы}

В ходе выполнения пятой лабораторной работы по курсу «Дискретный анализ» я изучил различные подходы к поиску паттернов в тексте, такие как использование суффиксного дерева и суффиксного массива. Суффиксное дерево оказывается особенно полезным, когда текст остаётся неизменным, а для поиска регулярно поступают новые паттерны. Однако этот метод требует много памяти, что привело к разработке суффиксного массива. Суффиксный массив выполняет те же задачи, но использует {O(n)} памяти, где {n} — длина текста.


\end{document}
