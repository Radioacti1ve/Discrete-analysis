\documentclass[12pt]{article}

\usepackage{fullpage}
\usepackage{multicol,multirow}
\usepackage{tabularx}
\usepackage{listings}
\usepackage{pgfplots}
\usepackage[utf8]{inputenc}
\usepackage[russian]{babel}
\usepackage{pgfplots}
\usepackage{tikz}

% Оригиналный шаблон: http://k806.ru/dalabs/da-report-template-2012.tex

\begin{document}

\section*{Лабораторная работа №7\, по курсу дискрeтного анализа: Жадные алгоритмы}

Выполнил студент группы М8О-312Б-22 МАИ \textit{Корнев Максим}.

\subsection*{Условие}
\textbf{Вариант:} 5. Оптимальная сортировка чисел

Дана последовательность длины N из целых чисел 1, 2, 3. Необходимо найти минимальное количество обменов элементов последовательности, в результате которых последовательность стала бы отсортированной. 


\newpage
\subsection*{Метод решения}

В рамках задачи первым шагом я определяю, сколько раз встречается каждый из элементов (1, 2, 3), а также подсчитываю количество элементов, расположенных не на своих местах. Например, если на позиции, предназначенной для единицы, находится двойка, то такая позиция считается неправильной (поскольку последовательность должна быть отсортирована, то сначала идут все 1, затем 2, а после них 3).

Затем я ищу пары элементов, которые находятся не на своих местах, и стараюсь минимизировать количество обменов между этими парами. Например, если в позиции для двоек стоят две единицы, а в позиции для единиц — три двойки, то минимально можно произвести 2 обмена (переставить местами две пары элементов). После проведения минимальных парных обменов могут остаться элементы, которые все еще находятся не на своих местах. Жадный алгоритм группирует такие оставшиеся элементы по тройкам, где в каждой группе присутствует по одному элементу каждого типа (1, 2, 3). Для перестановки трех элементов в группе требуется 2 обмена (мы можем поменять местами две любые пары).

% \newpage
\subsection*{Описание программы}

Для реализации алгоритма были реализованы следующая функция:
\begin{itemize}
    \item \texttt{функция minSwapsToSort}: 
    
    Функция вычисляет минимальное количество обменов, необходимых для сортировки массива, содержащего три типа элементов (1, 2, 3). Она находит элементы, которые находятся не на своих местах, минимизирует обмены между ними и возвращает минимальное количество обменов.

\end{itemize}

\newpage
\subsection*{Дневник отладки}

\begin{enumerate}
    \item Программа выдала {OK} с 1 попытки.
\end{enumerate}

\newpage
\subsection*{Тест производительности}

Алгоритм работает за время $O(n)$, где n - длина последовательности

\begin{tikzpicture}

\begin{axis}[xlabel={Длина последовательности}, ylabel={Время, мс}]
\addplot coordinates {
    (1000, 0.11)
    (2500, 0.3)
    (5000, 0.6)
    (10000, 1.1)
    (20000, 1.9)
};
\end{axis}
\end{tikzpicture}


\newpage
\subsection*{Выводы}


Я написал программу, которая находит минимальное количество обменов элементов в последовательности, состоящей из цифр 1, 2 и 3, чтобы отсортировать её. В процессе работы я изучил и применил жадный алгоритм — метод оптимизации. Такой подход позволил добиться сложности О(n), что является отличным результатом.



\end{document}


